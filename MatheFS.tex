% Mathe Formelsammlung für HM1 SoSe 2011
% 2 Seiten

% Dokumenteinstellungen
% ======================================================================	

% Dokumentklasse (Schriftgröße 6, DIN A4, Artikel)
\documentclass[6pt,a4paper]{scrartcl}

% Pakete laden
\usepackage[utf8]{inputenc}		% Zeichenkodierung: UTF-8 (für Umlautge)   
\usepackage[german]{babel}		% Deutsche Sprache
\usepackage{multicol}			% Spaltenpaket
\usepackage{amsmath}
\usepackage{amssymb}
\usepackage{esint}				% erweiterte Integralsymbole
\usepackage{multicol}			% ermöglicht Seitenspalten  
\usepackage{wasysym}			% Blitz
\usepackage{graphicx}
\usepackage{gensymb}
\usepackage{svg}

% Seitenlayout und Ränder:
\usepackage{geometry}
\geometry{a4paper, landscape, left=6mm,right=6mm, top=6mm, bottom=6mm} 
	
% Schriftart SANS für bessere Lesbarkeit bei kleiner Schrift
\renewcommand{\familydefault}{\sfdefault} 


% Custom Commands
\renewcommand{\thesubsection}{\arabic{subsection}}
\newcommand{\me}[1]{\ensuremath{\left\{#1\right\}}}
\newcommand{\dme}[2]{\ensuremath{\left\{#1\,\vert\,#2 \right\}}}
\newcommand{\abs}[1]{\ensuremath{\left\vert#1\right\vert}}
\newcommand{\un}[1]{\; \unit{#1} }
\newcommand{\unf}[2]{\;\left[ \unitfrac{#1}{#2} \right]}
\newcommand{\norm}[2][\relax]{\ifx#1\relax \ensuremath{\left\Vert#2\right\Vert}\else \ensuremath{\left\Vert#2\right\Vert_{#1}}\fi}
\newcommand{\enbrace}[1]{\ensuremath{\left(#1\right)}}
\newcommand{\nira}[1]{\ensuremath{\overset{n \rightarrow \infty}{\longrightarrow}}}
\newcommand{\os}[2]{\ensuremath{\overset{#1}{#2}}}
\makeatletter
\newcommand{\Ra}[0]{\ensuremath{\Rightarrow}}
\newcommand{\ra}[0]{\ensuremath{\rightarrow}}
\newcommand{\gk}[1]{\ensuremath{\left\lfloor#1\right\rfloor}}
\newcommand{\sprod}[2]{\ensuremath{%
  \setbox0=\hbox{\ensuremath{#2}}
  \dimen@\ht0
  \advance\dimen@ by \dp0
  \left\langle #1\rule[-\dp0]{0pt}{\dimen@},#2\right\rangle}}
% Für Mengen
\newcommand{\N}{\ensuremath{\mathbb N}}
\newcommand{\R}{\ensuremath{\mathbb R}}
\newcommand{\C}{\ensuremath{\mathbb C}}
\newcommand{\Q}{\ensuremath{\mathbb Q}}
\newcommand{\Z}{\ensuremath{\mathbb Z}}

% Dokumentbeginn
% ======================================================================
\begin{document}
%\subsection{}
% ----------------------------------------------------------------------

% Aufteilung in Spalten
\begin{multicols}{4}
      
\subsection{Allgemeines} % (fold)
\label{sub:allgemeines}

Dreiecksungleichung \qquad \qquad \qquad
\begin{math}\begin{array}{l}
	\abs{x + y} \le \abs{x} + \abs{y} \\
	\abs{\abs{x}- \abs{y}} \le \abs{x-y} 
\end{array}\end{math} \\
Cauchy-Schwarz-Ungleichung: \qquad
\begin{math}\begin{array}{l}
\left| \sprod{x}{y} \right| \le \| x\| \cdot \| y\|
\end{array}\end{math} \\
\\
Arithmetische Summenformel \qquad
\begin{math}\begin{array}{l}
	\sum\limits_{k = 1}^{n}k = \frac{n (n+1)}{2}
\end{array}\end{math}  \\
\\
Geometrische Summenformel \qquad 
\begin{math}\begin{array}{l}
	\sum\limits_{k = 0}^{n}q^k = \frac{1 - q^{n+1}}{1-q}
\end{array}\end{math}\\
\\                                   
Bernoulli-Ungleichung \qquad \qquad \quad
\begin{math}\begin{array}{l}
	(1+a)^n \ge 1 + na
\end{array}\end{math}\\   
\\
Binomialkoeffizient \qquad \qquad \qquad
\begin{math}\begin{array}{l}
	\binom{n}{k} = \frac{n!}{k!(n-k)!}  \\
	\binom{n}{0} = \binom{n}{n} = 1
\end{array}\end{math}\\ 
\\                   
Binomische Formel \qquad \qquad \qquad 
\begin{math}\begin{array}{l}
	(a+b)^n = \sum\limits_{k = 0}^{n} \binom{n}{k} a^{n-k} b^{k}
\end{array}\end{math}   \\ 
\\
Logarithmus \qquad \qquad \qquad \qquad \quad
\begin{math}\begin{array}{l}
	\ln{x^k}=k \cdot \ln{x}
\end{array}\end{math}   \\ 
Exponentialfunktion \qquad \qquad \qquad
\begin{math}\begin{array}{l}
	\exp(x) = \sum\limits_{n = 0}^\infty \frac{x^n}{n!}
\end{array}\end{math}   \\ 
\\
Wichtige Zahlen: $\sqrt{2} = 1,41421$\quad $e = 2,71828$ \quad $\pi =  3,14159$

\paragraph{Fakultäten} % (fold)
\label{par:fakultaeten}
$n! = 1 \cdot 2 \cdot 3 \cdot \ldots \cdot n$ \qquad  $0! = 1! = 1$ 
		

% paragraph fakultäten (end)
% subsubsection subsection_name (end)
% subsection allgemeines (end)


\subsection{Mengen}
% ----------------------------------------------------------------------

Eine Zusammenfassung wohlunterschiedener Elemente zu einer Menge\\
explizite Angabe: $A=\{1;2;3\}$\\
Angabe durch Eigenschaft: $A=\lbrace n \in \N \vert 0<n<4 \rbrace$\\
\subsubsection{Für alle Mengen A,B,C gilt:}
\begin{enumerate}\itemsep-1pt
\item $\emptyset \subseteq B $
\item $A \setminus (B \cup C) = (A \setminus B) \cap (A \setminus C)$
\item $(A \cap B) \cap C = A \cap (B \cap C)$\\
	$(A \cup B) \cup C = A \cup (B \cup C)$
\item $A \cap (B \cup C) = (A \cap B) \cup (A \cap C) \\
	A \cup (B \cap C) = (A \cup B) \cap (A \cup C)$
\end{enumerate}

\subsubsection{Offenheit}
Offen: \begin{enumerate}\itemsep-1pt
\item $\forall x \in X ~ \exists \delta > 0 : B_\delta(x)<X$
\item $\exists x \in \delta (\R^n\setminus X)$ mit $x \in (\R^n\setminus X)$
\end{enumerate}
Geschlossen: $\exists x \in \delta (\R^n\setminus X)$ mit $x \notin (\R^n\setminus X)$ \\
$\emptyset$ : beides \\

\subsubsection{Rationale Zahlen}
$\mathbb Q=\{\frac{p}{q}\ \vert\ p\in\mathbb Z; q\in\N\}$\\
\\
Jede rationale Zahl $\frac m n \in \mathbb Q$ hat ein Dezimaldarstellung.\\
$0,25\overline{54} =: a \rightarrow 10000a - 100a = 2554 -25 \Rightarrow a(9900) = 2529 \qquad \Rightarrow a = \frac{2529}{9900} = \frac{281}{1100}$

\subsection{Vollständige Induktion}
Behauptung: $f(n)=g(n)$ für $n_0 \le n \in \N$\\ 
IA: $n=n_0$: \quad Zeige $f(n_0)=g(n_0)=$wahr.\\
IV: Behauptung gilt für ein beliebiges $n\in\N$ \quad (Sei $f(n)=$wahr)\\
IS: $n \rightarrow n+1$: \quad Zeige $f(n+1)=\underset{=wahr}{f(n)}  \dotsc=g(n+1)$

\subsection{Komplexe Zahlen}
% ----------------------------------------------------------------------
Eine komplexe Zahl $z=a+b\mathbf{i},\ z\in \mathbb C a,b \in \R$ besteht aus einem Realteil $\Re(z)=a$ und einem Imaginärteil $\Im(z)=b$, wobei $\mathbf{i}=\sqrt{-1}$ die immaginären Einheit ist.
Es gilt: \quad $i^2 = -1$ \quad $i^4 = 1$ \\
\subsubsection{Kartesische Koordinaten}
Rechenregeln:\\
$z_1+z_2=(a_1+a_2)+(b_1+b_2)\mathbf{i}$\\
$z_1\cdot z_2=(a_1\cdot a_2-b_1\cdot b_2)+(a_1\cdot b_2+a_2\cdot b_1)\mathbf{i}$\\
\\
Konjugiertes Element von $z=a+b\mathbf{i}$:\\
$\overline{z}=a-b\mathbf{i}$\qquad \qquad \qquad \qquad \qquad \qquad \qquad \qquad $e^{\overline{ix}} = e^{-ix}$  \\
$z\overline{z}=|z|^2=a^2+b^2$\\
\\
Inverses Element:\\
$z^{-1}=\frac{\overline z}{\overline z z}=\frac{\overline z}{a^2+b^2}=\frac{a}{a^2+b^2} - \frac{b}{a^2+b^2}\mathbf{i}$ \\
\\
neutrales Element, zum Gleichungen lösen u.ä.: \\ 
$\frac{\overline{z}}{\overline{z}}=1$ \\
Bsp.: $z=\frac{x}{a+b i}=\frac{x}{a+bi}\cdot\frac{a-bi}{a-bi}=\frac{x\cdot(a-bi)}{a^2+b^2}$

\subsubsection{Polarkoordinaten}
$z=a+b\mathbf{i}\ne0$\ in Polarkoordinaten:\\
$z=r (\cos(\varphi)+\mathbf{i}\sin(\varphi))=r\cdot e^{\varphi \mathbf{i}}$\\
$r=|z|=\sqrt{a^2+b^2}\quad\varphi=\arg(z)=\begin{cases}+\arccos \left( \frac{a}{r}\right),  & b\ge0   \\  -\arccos \left( \frac{a}{r}\right), & b<0\end{cases}$ \\
Exponentialfunktion: $r\cdot e^{\varphi \mathbf{i}}=r (\cos(\varphi)+\mathbf{i}\sin(\varphi))$\\

\begin{description}\itemsep0pt
\item[Multiplikation:] $z_1\cdot z_2=r_1 * r_2 ( \cos ( \varphi_1 + \varphi_2) + \mathbf{i} \sin (\varphi_1 + \varphi_2))$
\item[Division:] $\frac{z_1}{z_2}=\frac{r_1}{r_2} ( \cos ( \varphi_1 - \varphi_2) + \mathbf{i} \sin (\varphi_1 - \varphi_2))$
\item[n-te Potenz:] $z^n=r^n\cdot e^{n\varphi \mathbf{i}}= r^n (\cos (n \varphi) + \mathbf{i} \sin (n \varphi))$
\item[n-te Wurzel:] $\sqrt[n]{z}= z_k = \sqrt[n]{r} \left(\cos \left(\frac{\varphi + 2k\pi}{n}\right) + \mathbf{i} \sin \left(\frac{\varphi + 2k\pi}{n}\right)\right) \\ k =0,1, \ldots, n-1$
\item[Logarithmus:] $\ln(z)=\ln(r) + \mathbf{i}(\varphi + 2k\pi)$ \quad (Nicht eindeutig!)
\end{description}
Anmerkung: Addition in kartesische Koordinaten umrechnen(leichter)!

\subsection{Matrizen}
% ----------------------------------------------------------------------
Eine Matrix ist eine Tabelle aus mathematischen Objekten.
Die Matrix $A=(a_{ij}) \in \mathbb K^{m\times n}$ hat $m$ Zeilen mit Index $i$ und $n$ Spalten mit Index $j$

\subsubsection{Allgemeine Rechenregeln}
\textbf{Merke:} Zeile vor Spalte! (Multiplikation, Indexreihenfolge, etc...)\\

\begin{tabular}{ll}	
	1)  $A+0=A$ & 2)  $1 \cdot A=A$ \\
	3)  $A+B=B+A$ & 4) $A \cdot B \ne B \cdot A$ (im allg.) \\
	5)  $(A+B)+C=A+(B+C)$ & 6) $\lambda (A+B) = \lambda A + \lambda B$\\ 
\end{tabular}
Multiplikation von $A\in \mathbb K^{m\times r}$ und $B\in \mathbb K^{r\times n}$: $AB\in\mathbb K^{m\times n}$

\subsubsection{Transponieren}
Falls $A=(a_{ij})\ \in \mathbb K^{m\times n}$ gilt: $A^\top=(a_{ji})\ \in \mathbb K^{n\times m}$\\
Regeln:\\
$(A+B)^\top=A^\top+B^\top$\qquad $(A\cdot B)^\top=B^\top\cdot A^\top$\qquad \\ $(\lambda A)^\top=\lambda A^\top$ \qquad $(A^\top)^\top=A$\\
\\
$A\in \mathbb K^{n\times n}$ ist symmetrisch, falls $A=A^\top$\qquad ($\Rightarrow$ diagbar)\\
$A\in \mathbb K^{n\times n}$ ist schiefsymmetrisch, falls $A=-A^\top$\\
$A\in \mathbb K^{n\times n}$ ist orthogonal(Spaltenvektoren=OGB), falls:\\
\qquad\ $AA^\top=E_n$\qquad $A^\top=A^{-1}$\qquad $\det A=\pm 1$\\
$A\in \mathbb C^{n\times n}$ ist hermitesch, falls $A=\overline{A}^\top$  \quad (kmplx. konj. u. transp.)


\subsubsection{Inverse Matrix $A^{-1}\in \mathbb K^{n\times n}$}
für die inverse Matrix $A^{-1}$ von $A\in \mathbb K^{n\times n}$ gilt: $A^{-1}A=E_n$\\
$(A^{-1})^{-1}=A$ \qquad $(AB)^{-1}=B^{-1}A^{-1}$ \qquad $(A^\top)^{-1}=(A^{-1})^\top$\\
\\
$A\ \in \mathbb R^{n\times n}$ ist invertierbar, falls: $\det (A) \ne 0 \quad \lor \quad rg(A)=n$\\
\\
Berechnen von $A^{-1}$ nach Gauß:\\
$AA^{-1}=E_n\quad\Rightarrow\quad (A|E_n)\overset{EZF}{\longrightarrow}(E_n|A^{-1})$\\

\subsubsection{Elementare Zeilen/Spaltenumformungen(EZF/ESF)}
$A \in \mathbb K^{m\times n}$ hat $m$ Zeilen $z_i\in \mathbb K^n$ und $n$ Spalten $s_j\in \mathbb K^m$
\begin{itemize}\itemsep0pt
\item \textbf{Addition} ($\lambda\ne 0$):\quad $\lambda_1 z_1 + \lambda_2  z_2$ \quad / \quad $\lambda_1  s_1 + \lambda_2 s_2$
\item Vertauschen von Zeilen/Spalten
\item Multiplikation mit $\lambda\ne 0$: \quad $\lambda \cdot z$ \quad  / \quad  $\lambda \cdot s$
\end{itemize}

\subsubsection{Rang einer Matrix $A$}
$A\in \mathbb K^{m\times n}$ mit $r$ lin. unabhängige Zeilen und $l$ "Nullzeilen":\\
Rang von $A$: $\mathrm{rg}(A)=m-l=r$\\  
Vorgehensweise: \\
\textbf{Zeilenrang (A):} Bringe $A$ auf ZSF $\Ra$ Zeilenrang$(A) = rg(A)$\\     
\textbf{Zeilenraum (A):}  $Z_A = \text{ Zeilen ungleich } 0$            \\
\textbf{Spaltenrang:} Bringe Matrix auf Spaltenstufenform        \\
\textbf{Kern:  }   $\ker(A) = \dme{x \in \mathbb R^n}{Ax= 0}$ \qquad $\mathrm{dim}(\ker(A))=n-r$ \\
\textbf{Bild: } $A^T \Ra EZF \Ra $ Zeilen $(\not= 0)$ bilden die Basis vom Bild. Die (lin. unabhängigen) Spalten von $A$ bilden eine Basis vom Bild.
\subsubsection{Lineares Gleichungssystem LGS}
Das LGS $Ax=b$ kurz $(A|b)$ mit $A\in \mathbb K^{m\times n}$, $x\in \mathbb K^n$, $b\in \mathbb K^m$ hat $m$ Gleichungen und $n$ Unbekannte.\\
\\
\textbf{Lösbarkeitskriterium:}\\
Ein LGS $(A|b)$ ist genau dann lösbar, wenn: $\mathrm{rg}(A)=\mathrm{rg}(A|b))$\\
Die Lösung des LGS $(A|b)$ hat $\dim{\ker A} = n-\mathrm{rg}(A)$ frei wählbare Parameter.\\
\\
Das homogene LGS: $(A|0)$ hat stets die triviale Lösung $0$\\
Das LGS hat eine Lsg. wenn $\det A \not= 0$ \quad $\rightarrow \exists A^{-1}$ \\
Summen und Vielfache der Lösungen von $(A|0)$ sind wieder Lösungen.

\subsubsection{Determinante von $A\in \mathbb K^{n\times n}$: $\det(A)=|A|$}

\begin{itemize}\itemsep0pt
\item $\det\begin{pmatrix}A&0\\C&D\end{pmatrix}=\det\begin{pmatrix}A&B\\0&D\end{pmatrix}=\det(A)\cdot\det(D)$
\item $\begin{vmatrix}\lambda_1&&* \\ &\ddots& \\ 0&&\lambda_n \end{vmatrix} = \lambda_1\cdot \ldots\cdot \lambda_n = \begin{vmatrix} \lambda_1&&0  \\  &\ddots& \\  *&&\lambda_n \end{vmatrix}$
\item $A=B \cdot C \quad \Rightarrow \quad |A|=|B| \cdot |C|$
\item $\det(A)=\det(A^\top)$
\item Hat $A$ zwei gleiche Zeilen/Spalten $\Rightarrow |A|=0$
\item $|A|=\sum\limits_{i=1}^n (-1)^{i+j} \cdot a_{ij} \cdot |A_{ij}|$ \qquad Entwcklng. n. $i$ter Zeile.
\item $\det(\lambda A)=\lambda^n \det(A)$
\item Ist $A$ invertierbar, so gilt: $\det(A^{-1})=(\det(A))^{-1}$
\item Vertauschen von Zeilen/Spalten ändert Vorzeichen von $|A|$
\item $\det(AB) = \det(A) \det(B) = \det(B) \det(A) = \det(BA)$
\end{itemize}
Vereinfachung für Spezialfall $A\in \mathbb K^{2\times 2}$:\\
$A=\begin{pmatrix}a&b\\c&d\end{pmatrix}\in \mathrm{K}^{2\times 2}\Rightarrow \det(A)=|A|=ad-bc$

\subsection{Vektorräume}
% --------------------------------------------------------------
Eine nichtleere Menge V mit zwei Verknüpfungen $+$ und $\cdot$ heißt $K$-Vektorraum über dem Körper $\mathbb K$.\\
\textbf{Linear Unabhängig:}
Vektoren heißen linear unabhängig, wenn aus: \\
$\lambda_1 \vec v_1 + \lambda_2 \vec v_2 + \ldots + \lambda_n \vec v_n = \vec 0$ folgt, dass $\lambda_1 = \lambda_2 = \lambda_n = 0$
\subsubsection{Skalarprodukt $\langle v,w \rangle$} 
	\begin{description}
	\item[Bilinear:] $\langle \lambda v+v',w \rangle=\lambda\cdot\langle v,w \rangle + \langle v',w \rangle$
	\item[Symmetrisch:] $\langle v,w \rangle=\langle w,v \rangle$
	\item[Positiv definit:] $\langle v,v \rangle\ge0$ 
	\item[Im kartesischen Koordinatensystem:] $\langle v,w \rangle=v_1 w_1+…+v_n w_n$
	\end{description}  
Skalarprodukt bezüglich \textbf{symmetrischer, quadratischer} und \textbf{positiv definite} Matrix $A\in \mathbb R^{n\times n}$\\
$\langle v,w \rangle_A=v^T A w$\\
Matrix A positiv definit falls $\det (a_{11}) > 0 \land \det \left(\begin{matrix} a_11 & a_12\\ a_21 & a_22\end{matrix}\right) > 0 \land \dotsc \land \det (A)>0$   \\
\textbf{Orthogonale Projektion} $p\in U^n$ von $q\in V^m$ auf $\sum u_i$:
\begin{eqnarray*}
   p=\sum_{i=1}^n \sprod{q}{\frac{u_i}{\abs{u_i}}}\frac{u_i}{\abs{u_i}} \quad = q - p^\perp
\end{eqnarray*} 
\textbf{Winkel} \quad 	$\sprod{\vec a}{\vec b} = a \cdot b \cdot \cos \phi$ \qquad
$\phi = \arccos \enbrace{ \frac{\sprod{x}{y} }{\norm{x} \norm{y} } }$\\
\textbf{Polynome} $<p(x),q(x)>=\int\limits_{0}^{1}p(x)q(x)\,dx$

\subsubsection{Betrag von Vektoren}
$
||\vec a||=\sqrt{<\vec a,\vec a>} =\sqrt{a_1^2+a_2^2+\ldots +a_n^2}
$


\subsubsection{Orthogonalität}
\textbf{Orthonormalisierungsvefahren von $n$ Vektoren nach Gram-Schmidt:}\\
1. $b_1=\frac{v_1}{\|v_1\|}$ \qquad (Vektor mit vielen 0en oder 1en)\\
2. $b_{k+1}= \frac{b_{k+1}^{'}}{\|b_{k+1}^{'}\|}$\ \ mit \ \ $b_{k+1}^{'}=v_{k+1}-\sum_{i=1}^k \langle v_{k+1},b_i \rangle \cdot b_i$\\
\\
\textbf{Ausgleichsrechnung:}\\
Experiment: $(t_1,y_1), \hdots, (t_n,y_n)$\\
$f_1: \mathbb R \rightarrow \mathbb R, f_1(x) =1$ \qquad $f_2: \mathbb R \rightarrow \mathbb R, f_2(x) = x$
\begin{eqnarray*}
\Rightarrow A = \begin{pmatrix}f_1(t_1) & f_2(t_1)\\\vdots & \vdots\\f_1(t_n) & f_2(t_n)\end{pmatrix} \qquad v = \begin{pmatrix}y_1\\\vdots\\y_n\end{pmatrix}
\end{eqnarray*}
$A^{\top}Ax=A^{\top}v \rightarrow $ LGS lösen nach x\\
$f: \mathbb R \rightarrow \mathbb R, f(x) = x_1 f_1(x) + \hdots + x_n f_n(x)$
\\
\textbf{Orthogonale Projektion in UVR:} \quad \\
1. Normiere Basis von $U$. \\
2. $u = \sprod{b_1}{v}b_1 + \sprod{b_2}{v}b_2 \ldots \Ra u^\perp = v - u$ \\
Abstand von $v$ zu $U$: $\norm{u^{\perp}}$

\subsubsection{Vektorprodukt}
$\vec a\times\vec b=\left( \begin{matrix} a_2b_3-a_3b_2\\a_3b_1-a_1b_3\\a_1b_2-a_2b_1\end{matrix}\right)$\qquad $\vec a,\vec b\ \in \mathbb R^3$\\
\\
$\vec a\times\vec b \perp \vec a,\vec b$ \qquad ($\vec a\times\vec b=0\ \Leftrightarrow\ \vec a;\vec b$\ linear abhängig.\\
$||\vec a\times\vec b||=||\vec a||\cdot||\vec b||\cdot \sin\left(\measuredangle (\vec a;\vec b)\right)\mathrel{\widehat{=}}$\ Fläche des Parallelogramms\\
Graßmann-Identität: $\vec a\times(\vec b \times \vec c)\equiv\vec b\cdot(\vec a \cdot \vec c)-\vec c\cdot(\vec a \cdot \vec b)$\\
\\
Spatprodukt:\\
$[a,b,c]:=\langle \vec a\times\vec b,\vec c\rangle=\det (a,b,c)\mathrel{\widehat{=}}$\ Volumen des Spates.\\
$[a,b,c]>0\ \Rightarrow\ a,b,c$\ bilden Rechtssystem \\ $[a,b,c]=0\ \Rightarrow\ a,b,c$\ linear abhängig\\ \\
Orthogonale Zerlegung eine Vektors v längs a:\\
$v = v_a + v_{a^\perp} \text{ mit } v_a = \frac{\sprod{v}{a} }{\sprod{a}{a} }*a \text{ und }	 v_{a^\perp} = v - v_a	$
\subsubsection{Basis (Jeder VR besitzt eine Basis!)} % (fold)
\label{sub:basis}
 Eine Teilmenge $B$ heißt Basis, von $V$ wenn gilt:
\begin{itemize}\itemsep0pt
	\item $\left\langle B \right\rangle =V$  $B$ erzeugt $V$
	\item $B$ ist linear unabhängig
\end{itemize}                   


% subsection basis (end)
\subsubsection{Dimension} % (fold)
\label{sub:dimension}
  \begin{eqnarray*}
  	   n:= \abs{B} \in \mathbb N_0 \text{ Dimension von }V \quad & \quad \dim (V) = n
  \end{eqnarray*}   
Mehr als $n$ Vektoren sind stehts linear abhängig. \\
Für jeden UVR $U \subset V$ gilt: $\dim (U) < \dim (V)$ 
% subsection dimension (end)






\subsection{Untervektorräume}
Eine Teilmenge $U$ eines $K-$Vektorraums $V$ heißt Untervektorraum (U-VR) von $V$, falls gilt:
\begin{enumerate}\itemsep0pt
\item $U\neq \varnothing$ \qquad ($0\in U$)
\item $u+v\in U \quad \forall u,v\in U$
\item $\lambda u \in U \qquad \forall u\in U,\forall \lambda \in K$
\end{enumerate}
Wegen (3.) enthält ein UVR $U$ stets den Nullvektor $0$. Daher zeigt man (1.) meist, indem man $0\in U$ nachweist.\\
\\
\textbf{Triviale UVR}: $U=\{0\}$ mit $B = \emptyset$ \qquad $U=V$ mit $B_U=B_V$

\subsection{Sachen die man in der Schule hätte lernen sollen}
Mitternachtsformel (abc): \\
$a x^2 + bx + c = 0 \\
\rightarrow x_{1/2} = \frac{-b \pm \sqrt{b^2 - 4ac}}{2a}$ \\
Andere Mitternachtsformel (pq): \\
$x^2+p x+q=0 \\
\rightarrow x_{1/2} = -\frac{p}{2} \pm \sqrt{\enbrace{\frac{p}{2}}^2-q}$ \\
Doppelbruch: $\frac{\frac{a}{b}}{\frac{c}{d}}=\frac{a d}{b c}$ \\
Binomische Formeln:
\begin{enumerate}\itemsep-1pt
\item $(a+b)^2=a^2+2ab+b^2$
\item $(a-b)^2=a^2-2ab+b^2$
\item $(a+b)(a-b)=a^2+b^2$
\item $(a + b)^n = \sum_{k=0}^{n} \binom{n}{k} a^{n-k} \cdot b^k, n \in \N$
\item $(a \pm b)^3 = a^3 \pm 3 a^2 b + 3 a b^2 \pm b^3$
\end{enumerate}
Binomialkoeffizient: $\binom{n}{k} = \frac{n!}{(n-k)! k!}$ \\
Exponentialfunktion: $\exp(x) = \sum\limits_{n = 0}^\infty \frac{x^n}{n!}$ \\
\end{multicols}
% Ende der Spalten


% Dokumentende
% ======================================================================
\end{document}
